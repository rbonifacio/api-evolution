\documentclass{article}

\usepackage{enumerate}
\usepackage{xspace}

\newcommand{\report}{report\xspace}

\author{Handrick Costa and Rodrigo Bonif\'{a}cio and Stefan Krueger and Eric Bodden}

\title{On the Design and Evolution of Cryptographic APIs}

\begin{document}

\maketitle

\begin{abstract}

\end{abstract}

\section{Introduction}

% discuss the challenge of design resilient APIs.

Application Programming Interfaces (APIs) are essencial
building blocks for software development and are widespread
used for reusing components (in many cases even accross applications
from different domains). A properly designed API should provide reusable
abstractions and hide implementation details, so that the API could
evolve while keeping backward compatibility. In this way, 
programs that depend on a specific API do not break
with the evolution of such an API. Surely, designing a
resilient API is a challenging task, and several research works
try to characterize how APIs evolve.   

Differently from existing works, in this \report we focus on
the design and evolution of cryptographic APIs. We have two
main reasons for that. First, we have developed a domain specific
language (CrySL) for specifying the correct usage of cryptographic APIs.
Depending on the evolution of these APIs, new constructs should
be necessary to this language. Second, the discovery of
threats related to cryptographic algorithms might
introduce new vulnerabilities into programs, which might obligate
developers to evolve an API---since insecure implementations
should be removed from new releases or at least marked as deprecated.
This leads to a conflicting situation:
removing insecure implementations of
a cryptographic API might lead to \emph{breaking changes}, though
keeping insecure implementations available in APIs
might guide developers to implement insecure software.


In this way, we investigate the following research
questions:

\begin{enumerate}[(RQ1)]
\item What idioms do developers use to design cryptographic APIs?

\item How do developers evolve cryptographic API and how often breaking changes occur?
  
\item Do developers depracate old and insecure algorithms of cryptographic APIs? 
\end{enumerate}  

Answer to these questions might bring new insights about how to design and use cryptographic APIs and how to specify policies and static analysis tools that aim
to guarantee the correct usage of these APIs, considering that they
might evolve along the time. Altogether, the contribution of
this report is three-fold:



% It seems that the old style for implementing modules in
% Java projects is not suitable for clearly separating
% internal utility functions of the real API classes. 
\end{document}
